% !TeX spellcheck = en_GB
\section{Modeling}
\subsection{Introduction}
The system is modeled as \textbf{N Transmitter-Receiver couples} which communicate through \textbf{C Channels}. A \textit{collision} can occur on a \textbf{Channel} if more than one \textbf{Transmitter} want to transmit a packet in that \textbf{Channel}. The \textbf{Transmitter} stores in a queue the packets that it wants to transmit and, then, it sends them; the \textbf{Channel} "knows" if a collision occurs and handles it; the \textbf{Receiver} only receive packets.  
\subsection{General Assumptions}
The following general assumptions have been made:
\begin{itemize}
	\item \textbf{Slotted}: packets are attempted to be \textbf{Transmitter} by the Transmitter only at the \textbf{beginning of the time-slot}
	\item \textbf{Constant Packet Size and Transmission Rate}: each packet has a \textbf{constant packet size} and each \textbf{Transmitter} has a constant and equal transmission rate for which to transmit a packet (without collision) from the \textbf{Receiver} to the transmitter will last \textbf{one time-slot}.
	\item \textbf{No Propagation Error in the channel}: the only cause of a failed transmission has to be considered as the \textit{packet} collision. Other causes, (i.e. path-loss, shadowing, small-scale fading), are neglected. 
	\item \textbf{FIFO Queues of unlimited Capacity at the Transmitter} 
	\item \textbf{Transmitters and Receivers always synchronized with the time-slot period}: the \textbf{Receiver} knows in which \textbf{Channel} the \textbf{Transmitter} will try to send his packet in each time-slot and the \textbf{Receiver} will be ready to listen in the correct \textbf{Channel}.
	\item \textbf{After an eventual collision the transmitter will change his channel choice}
\end{itemize}

\subsection{Preliminar Validation}
Before the implementation a preliminar validation phase is necessary to ensure that the model is correct. Let analyze if the assumptions made in the previous section are reasonable:
\begin{itemize}
	\item The \textbf{slotted assumption} is reasonable due to the fact that exist some network protocols which work under this assumption, see for instance Slotted Aloha that is the most famous one.
	\item We consider the \textbf{packet size constant} because if the packet length is so large that more than one slots are needed, we can consider, from the viewpoint of the model, that this unique packet send in two different slots is like two packets of fixed-length send each one in a slot. Taken this into account is reasonable to state that the packet size is constant and that one packet is transmitted within a slot (if no collisions occur).
	\item The critical issue of every slotted network is the one related to the \textbf{collisions}: they have an huge impact on the general performance of the network, so it is reasonable to neglect the other propagation errors that are not network-specific or that depend from the environment (as path-loss).
	\item It's reasonable that the transmitter and the receiver are \textbf{synchronized with the time-slot period}, in fact other slotted-based network as slotted ALOHA requires a synchronization of this type.
	\item When a packet collide it's reasonable to think that the transmitter will change the transmission channel in order to avoid another collision. Indeed this along with the back-off time are the techniques that should avoid another collision. 
\end{itemize}
\subsection{Factors}
The following factors have been defined which may affect the performance of the system:
\begin{itemize}
	\item \textbf{N}: \textbf{Transmitter-Receiver} Couples.
	\item \textbf{C}: numbers of \textbf{Channels}.
	\item \textbf{p}: \textbf{probability of success} for sending a packet in the current time-slot for a Transmitter.
	\item \textbf{$\dfrac{1}{\lambda}$}: exponential distribution \textbf{mean inter-arrival time of packets} at the Transmitter.
	\item $T_{slot}$: \textbf{time-slot duration}.
	
\end{itemize}